\documentclass{article}

\usepackage{graphicx}
\usepackage{hyperref}
\usepackage{fontawesome}
\hypersetup{
    colorlinks,
    citecolor=blue,
    filecolor=blue,
    linkcolor=blue,
    urlcolor=red
}
\usepackage{url}
% verbatim small
\makeatletter
\g@addto@macro\@verbatim\footnotesize
%\renewcommand{\l@section}{\@dottedtocline{1}{1.5em}{2.6em}}
\renewcommand{\l@subsection}{\@dottedtocline{2}{1.5em}{3.0em}}
\renewcommand{\l@subsubsection}{\@dottedtocline{3}{4.4em}{4.0em}}
\makeatother

\title{Tutorial Emacs Bahasa Indonesia}
\author{
  @kholidfu\footnote{\url{http://twitter.com/kholidfu}}\\
  \footnotesize{\url{https://github.com/kholidfu/emacs\_doc}}
}
\date{\today}

\begin{document}

\maketitle
\tableofcontents
\pagebreak

\section{Pendahuluan}
Dari sekian banyak aplikasi penyunting teks 'jadul' di keluarga
Unix adalah \verb=emacs=. Sama halnya dengan \verb=Vim=, penyunting
teks satu ini juga memiliki penggemar tersendiri (baca: \emph{fanbase} \faSmileO), yang kadang
karena kefanatikannya sampai terjadi \emph{war} antara \verb=emacs=
vs \verb=Vim=. Pertanyaan yang sering muncul adalah ``apakah
\verb=emacs= dapat melakukan ini, apakah \verb=Vim= memiliki fitur
ini?`` Dalam artikel ini, penulis tidak mau terjebak dalam 
\emph{war} tersebut, melainkan sekedar menuliskan pengalaman 
menggunakan \verb=emacs=, dari kacamata seorang awam.

Salah satu alasan saya belajar \verb=emacs= adalah kenyataan
bahwa beberapa \emph{key bindings} pada program di *Nix 
menggunakan \verb=emacs=, sebagai contoh \verb=Bash=. Untuk
melakukan navigasi perintah pada \verb=Bash=, 
\emph{default key bindings} yang digunakan adalah \verb=emacs=.

Sebagai contoh, Anda memiliki perintah seperti berikut:

\begin{verbatim}
$ ssh -i key.pem -D 9999 ubuntu@someserver.com
\end{verbatim}

Anda dapat menyunting perintah di atas dengan menggunakan
navigasi seperti pada \verb=emacs=, misal:

\begin{verbatim}
|----------+----------------------------|
| Perintah | Keterangan                 |
|----------+----------------------------|
| M-f      | Menuju ke kata selanjutnya |
| M-4 M-f  | Maju 4 kata                |
| C-a      | Menuju ke awal baris       |
| C-e      | Menuju ke akhir baris      |
|----------+----------------------------|
\end{verbatim}

Atau perintah \verb=C-r= untuk melihat \emph{history} perintah
pada \verb=Bash=, di mana perintah ini kalau di \verb=emacs=
digunakan untuk pencarian ke belakang (\emph{backward-search}),
hampir sama bukan?

Itu hanya contoh kecil saja, karena saya sering bekerja dengan
\verb=terminal=, maka menguasai dasar-dasar navigasi dalam
\verb=emacs= berguna untuk menyunting perintah secara cepat.

Sekali lagi, tulisan ini bukanlah sebuah tutorial lengkap 
program \verb=emacs=, melainkan sekedar catatan pribadi penulis
yang coba dituangkan kedalam sebuah berkas elektronik dengan 
tujuan agar lebih mudah disebarluaskan kepada siapa saja yang
membutuhkan. Semoga bermanfaat...

\subsection{Tentang Emacs}

\emph{GNU Emacs is an extensible, customizable text editor - and more.}\footnote{http://www.gnu.org/software/emacs/}

\verb=Emacs= termasuk dalam keluarga aplikasi penyunting teks,
namun memiliki kelebihan dari sisi \emph{extensibility}. Tidaklah
heran beberapa kalangan menyebut \verb=emacs= sebagai sebuah sistem
operasi, karena selain menyunting teks, dengan \verb=emacs=, Anda dapat
membaca \emph{email}, menjelajah \emph{website}, dan memainkan 
beberapa aplikasi permainan sederhana, \emph{FTP client}, \emph{news reader},
dan masih banyak lagi. Anda pun dapat membuat versi \verb=emacs= Anda sendiri,
selama Anda memahami bagaimana cara melakukan pengembangan pada \verb=emacs=.

Pengembangan aplikasi ini dimulai pada pertengahan 1970-an, dan 
masih aktif sampai saat tutorial ini ditulis (2013). \verb=Emacs=
memiliki lebih dari 2.000 perintah \emph{built-in} yang dapat
digunakan untuk membuat \emph{macro} untuk membuat pekerjaan
menjadi otomatis.

\verb=Emacs= dibuat menggunakan \verb=emacs lisp=, sebuah
\emph{variant} dari bahasa \verb=Lisp=.

Versi pertama dari \verb=emacs= ditulis oleh Richard Stallman dan
Guy L. Steele pada tahun 1976.

\verb=Emacs= sendiri memiliki banyak \emph{variant}, dan saat ini
\emph{variant} yang paling banyak dipakai adalah \verb=GNU Emacs=
yang dibuat oleh Stallman untuk \emph{GNU Project}.

Berikut adalah tampilan default emacs\footnote{GNU Emacs 25.2.2}.

\vspace{12pt}

\includegraphics[scale=0.45]{images/emacs1.png} 

\vspace{12pt}


Anda juga dapat menjalankan emacs dalam mode text-mode (non-GUI) 
dengan menambahkan parameter \verb=-nw=.

\begin{verbatim}
$ emacs -nw
\end{verbatim}

\vspace{12pt}

\includegraphics[scale=0.5]{images/emacs-nox.png} 

\vspace{12pt}

Dan berikut ini tampilan awal emacs yang sudah saya custom (GUI):

\vspace{12pt}

\includegraphics[scale=0.45]{images/emacs2.png} 

\vspace{12pt}

\section{Bekerja dengan Berkas}
Berikut perintah ketika bekerja dengan berkas:

\begin{verbatim}
|----------+--------------------------------------------------|
| Perintah | Keterangan                                       |
|----------+--------------------------------------------------|
| C-x C-f  | Membuka berkas                                   |
| C-x C-s  | Menyimpan berkas                                 |
| C-x s    | Menyimpan semua berkas yang sedang disunting     |
| C-x i    | Memasukkan teks dari berkas lain ke dalam berkas |
|          | yang sedang disunting                            |
| C-x C-v  | Mengganti berkas yang sedang disunting dengan    |
|          | berkas lain                                      |
| C-x C-w  | Menyimpan buffer ke dalam berkas lain            |
| C-x C-q  | Mengubah ke mode read-only                       |
| C-x C-k  | Menutup berkas                                   |
|----------+--------------------------------------------------|
\end{verbatim}

\section{Dasar-dasar Navigasi}
Sebelum masuk ke pembahasan mengenai navigasi, mari kita
pelajari terlebih dahulu apa yang dimaksud dengan karakter
\verb=Control= dan karakter \verb=Meta= pada \verb=emacs=.

Penting juga untuk diketahui penggunaan istilah \emph{point} yang tidak 
lain adalah \emph{cursor} (kursor) atau posisi letak \emph{mouse} Anda dalam
dokumen.

\subsection{Karakter Control dan Meta}
Program penyunting teks \verb=emacs= akan sering sekali
melibatkan dua tombol pada \emph{keyboard}, yakni tombol
\verb=Ctrl= dan tombol \verb=Meta=. Untuk tombol \verb=Ctrl=
(sering disingkat dengan \verb=C=), saya yakin semua sudah
mengetahui, lalu di mana letak tombol \verb=Meta (M)=?

Saya menggunakan sistem operasi \verb=Ubuntu= dan tombol
\verb=Meta= secara otomatis di-\emph{assign} pada tombol
\verb=Alt=.

Ketika Anda bekerja dengan \verb=emacs= dalam mode \emph{GUI},
penggunaan tombol \verb=Alt= sebagai karakter \verb=Meta= 
tidak menjadi masalah. Namun ketika Anda bekerja dengan mode
\emph{command line} (\verb=emacs -nw=) menggunakan program 
\verb=gnome-terminal=, hal ini akan menjadi masalah, karena 
begitu Anda tekan tombol \verb=Alt=, 
otomatis \verb=gnome-terminal= akan mengaktifkan \emph{menu access}.

Solusinya adalah dengan mengubah pengaturan pada 
\verb=gnome-terminal= dengan cara:

\begin{verbatim}
Edit->Keyboard Shortcuts->Hilangkan centang pada "Enable menu access keys"
\end{verbatim}

\subsection{Navigasi}

Sama seperti \verb=Vim=, pengguna \verb=emacs= akan
merasakan manfaat yang lebih besar ketika meminimalkan
penggunakan perangkat \emph{mouse} mereka. Untuk itu mari kita
belajar melakukan navigasi pada \verb=emacs= menggunakan
perangkat \emph{keyboard}.

Berikut ini beberapa perintah navigasi dasar pada \verb=emacs=:

\begin{verbatim}
|----------+------------------------------------------------|
| Perintah | Keterangan                                     |
|----------+------------------------------------------------|
| C-f      | Bergerak maju 1 karakter                       |
| C-b      | Bergerak mundur 1 karakter                     |
| M-f      | Bergerak maju 1 kata                           |
| M-b      | Bergerak mundur 1 kata                         |
| C-a      | Bergerak ke awal baris                         |
| C-e      | Bergerak ke akhir baris                        |
| C-n      | Bergerak ke baris selanjutnya (bawah)          |
| C-p      | Bergerak ke baris sebelumnya (atas)            |
| C-v      | Menggulung layar ke bawah                      |
| M-v      | Menggulung layar ke atas                       |
| C-l      | Menempatkan point tepat di tengah layar        |
| M-<      | Menuju ke awal (top) buffer                    |
| M->      | Menuju ke akhir (bottom) buffer                |
| M-a      | Menuju ke awal kalimat                         |
| M-e      | Menuju ke akhir kalimat                        |
| C-m      | Membuat baris baru tepat di bawah posisi point |
| C-o      | Membuat baris baru tepat di atas posisi point  |
| C-j      | Membuat baris baru dan indent                  |
|----------+------------------------------------------------|
\end{verbatim}

Selain, perintah di atas, salah satu favorit saya untuk melakukan navigasi 
adalah dengan memanfaatkan \emph{search}. \verb=Emacs= memiliki dua 
macam \emph{search}, yakni \emph{incremental search} dan 
\emph{non-incremental search}. Masing-masing terbagi lagi
menjadi dua, yakni \emph{forward-search} dan \emph{backward-search}.

Berikut perintah untuk melakukan \emph{search}:

\begin{verbatim}
|-------------------------------+---------------------------------|
| Perintah                      | Keterangan                      |
|-------------------------------+---------------------------------|
| C-s <character/word> <RETURN> | forward incremental search      |
| C-r <character/word> <RETURN> | backward incremental search     |
| C-s <RETURN> <character/word> | forward non-incremental search  |
| C-r <RETURN> <character/word> | backward non-incremental search |
|-------------------------------+---------------------------------|
\end{verbatim}

Berikut contoh untuk melakukan navigasi secara cepat menggunakan fitur
\emph{incremental search}:

\begin{verbatim}
emacs adalah program penyunting teks yang handal.
\end{verbatim}

Letakkan \emph{point} pada awal kalimat di atas dengan menekan \verb=C-a=, 
kemudian tekan \verb=C-s=, untuk menuju ke huruf \verb=t= pada kata 
\verb=teks=, gunakan perintah \verb=C-s t C-s= kemudian tekan \verb=enter=. 
Untuk menuju ke karakter \verb=t= berikutnya, cukup tekan \verb=C-s=
berulang-ulang.

Dan berikut contoh untuk \emph{backward incremental search}, mari kita
asumsikan \emph{point} berada pada akhir baris, untuk menuju ke huruf 
\verb=e= pada kata \verb=teks=, cukup tekan \verb=C-r e=, kemudian tekan 
\verb=enter=.

Dengan menguasai hal ini saja, penulis yakin kecepatan Anda dalam menyunting 
teks akan meningkat drastis (dibandingkan dengan harus menggunakan 
\emph{mouse}).

\subsection{Numeric argument}
\emph{Numeric argument} pada \verb=emacs= merupakan salah satu fitur
untuk meningkatkan efisiensi pekerjaan. Dengan fitur ini, kita dapat 
mengulang seberapa banyak sebuah perintah untuk dijalankan. Berikut 
beberapa contoh penggunaan \emph{numeric argument} pada \verb=emacs=.

\begin{verbatim}
|-----------+----------------------------------------|
| Perintah  | Keterangan                             |
|-----------+----------------------------------------|
| C-u 8 C-f | Bergerak 8 karakter ke kanan           |
| C-u 8 C-n | Bergerak 8 baris ke bawah              |
| C-u 8 *   | Mengetik tanda bintang sebanyak 8 kali |
|-----------+----------------------------------------|
\end{verbatim}

\section{Dasar-dasar Penyuntingan Teks}

Untuk memulai pengetikan, Anda tinggal menekan tombol-tombol yang ada pada
\emph{keyboard} Anda. Satu hal yang perlu ditekankan di sini, apabila Anda 
ingin mengetik secara efektif dan efisien, ada 2 syarat penting, yakni pertama 
tinggalkan penggunaan perangkat \emph{mouse} Anda, dan sebisa mungkin hindari 
penggunaan tombol panah untuk melakukan navigasi.

\subsection{Menyorot Teks}
Melakukan seleksi (sorot) pada \verb=emacs= cukup dengan menekan tombol:

\begin{verbatim}
|-----------+------------------------------------|
| Perintah  | Keterangan                         |
|-----------+------------------------------------|
| C-<Space> | Mark Set (mengaktifkan mode sorot) |
|-----------+------------------------------------|
\end{verbatim}

Kemudian diikuti dengan \emph{motion} untuk menandai daerah yang akan disorot.
Misal untuk menyorot sebuah baris:

\begin{verbatim}
|-----------+-----------------------|
| Perintah  | Keterangan            |
|-----------+-----------------------|
| C-a       | Menuju ke awal baris  |
| C-<Space> | Mulai sorot           |
| C-e       | Menuju ke akhir baris |
|-----------+-----------------------|
\end{verbatim}

\subsection{Menghapus Teks}

Berikut kombinasi perintah untuk melakukan penghapusan teks dalam \verb=emacs=:

\begin{verbatim}
|---------------+--------------------------------------------------|
| Perintah      | Keterangan                                       |
|---------------+--------------------------------------------------|
| <Backspace>   | Menghapus 1 karakter sebelum point               |
| C-d           | Menghapus 1 karakter di atas point               |
| M-<Backspace> | Menghapus 1 kata sebelum point                   |
| M-d           | Menghapus 1 kata setelah point                   |
| C-k           | Menghapus dari posisi point sampai akhir baris   |
| M-k           | Menghapus dari posisi point sampai akhir kalimat |
|---------------+--------------------------------------------------|
\end{verbatim}

Anda juga dapat menggabungkan perintah penghapusan ini dengan fitur
\emph{numeric argument}, misal untuk menghapus 2 baris, gunakan perintah 
berikut:

\begin{verbatim}
|-----------+----------------------------|
| Perintah  | Keterangan                 |
|-----------+----------------------------|
| C-a       | Menuju ke awal baris       |
| C-u 2 C-k | Menghapus 2 baris ke bawah |
|-----------+----------------------------|
\end{verbatim}

\subsection{Killing and Yanking}
\verb=Emacs= menggunakan istilah \emph{killing} untuk \emph{cutting}, dan
\emph{yanking} untuk \emph{pasting}. \emph{Killing} memindahkan teks dari
\emph{buffer} ke bagian paling atas dari \emph{kill ring} (\emph{clipboard}).
\emph{kill ring} ini sendiri mampu menampung sampai 60 buah \emph{killed items}.

Anda dapat memanggil \emph{kill ring} ini menggunakan perintah \verb=M-y=, 
tekan lagi untuk memanggil \emph{kill items} sebelumnya, dan seterusnya.

Berikut cara untuk berinteraksi dengan \emph{kill ring}:

\begin{verbatim}
|----------+-----------------------------------|
| Perintah | Keterangan                        |
|----------+-----------------------------------|
| C-w      | Killing                           |
| C-y      | Yanking                           |
| M-y      | Memanggil yank sebelumnya         |
| M-y      | Memanggil yank sebelumnya, dst... |
|----------+-----------------------------------|
\end{verbatim}

Membingungkan? Tenang ... Itu hanya karena belum terbiasa saja ... :)

\subsection{Salin dan Tempel}

Perintah untuk menyalin teks dalam \verb=emacs=:

\begin{verbatim}
|----------+----------------------|
| Perintah | Keterangan           |
|----------+----------------------|
| M-w      | Menyalin (copy) teks |
| C-y      | Paste (yank) teks    |
|----------+----------------------|
\end{verbatim}

Berikut ini contoh perintah untuk salin baris:

\begin{verbatim}
|----------+--------------------|
| Perintah | Keterangan         |
|----------+--------------------|
| C-a      | Menuju awal baris  |
| C-k      | Hapus baris        |
| C-y      | Tempel baris       |
| C-m      | Membuat baris baru |
| C-y      | Tempel lagi        |
|----------+--------------------|
\end{verbatim}

\subsection{Salin Tempel Seleksi}

\begin{verbatim}
|-----------------------+---------------------|
| Perintah              | Keterangan          |
|-----------------------+---------------------|
| C-<Space>             | Start the selection |
| C-p / C-n / C-f / C-b | Memilih area        |
| C-w                   | Cut                 |
| M-w                   | Copy                |
| C-y                   | Paste               |
|-----------------------+---------------------|
\end{verbatim}

\subsection{Memotong (Cut) Teks}

Perintah untuk memotong teks dalam \verb=emacs=:

\begin{verbatim}
|----------+-------------------------|
| Perintah | Keterangan              |
|----------+-------------------------|
| C-w      | Memotong (cut) teks     |
| C-y      | Menempel (paste) (yank) |
|----------+-------------------------|
\end{verbatim}

\section{Mengakses Bantuan (Help)}
Ini adalah fitur yang \textbf{wajib} dikuasai oleh setiap pengguna 
\verb=Emasc=, kenapa? Karena otak kita tidak mungkin untuk mengingat
semua kombinasi tombol perintah yang ada pada \verb=Emacs=.

Berikut ini beberapa tips yang penulis pakai untuk mengakses menu bantuan
(\emph{help}) untuk menemukan informasi yang kita inginkan.

Tekan \verb=C-h a= (\verb=a= untuk \verb=apropos=) untuk memunculkan 
menu bantuan, kemudian ketik \emph{keyword} yang ingin kita cari, 
misalkan saya ingin mengetahui bagaimana cara membuka daftar 
\emph{buffers} yang aktif, saya ketikkan saja \emph{buffers list}, 
kemudian \verb=emacs= merespon dengan informasi yang [mungkin] terkait 
dengan \emph{keyword buffers list}. Selanjutnya tinggal kita pilih dan
buka informasi mana yang mungkin relevan.

Contoh kedua adalah mencari tahu kombinasi tombol pada \verb=emacs= untuk
melakukan suatu pekerjaan. Misalkan saya ingin mencari kombinasi tombol untuk
menggulung layar pada \emph{window} lain. Ketikkan saja \verb=C-h b=,
\verb=b= di sini singkatan dari \emph{bindings}, ketik 
\emph{keyword scroll other}, tekan \verb=Enter=, dan muncullah informasi
yang mungkin terkait dengan apa yang kita cari, untuk lebih cepat menemukan
tinggal kita gunakan saja fungsi \emph{search} (\verb=C-s=) diikuti dengan
kata yang ingin dicari.

Dan tentunya masih banyak lagi contoh lain, yang jelas fitur bantuan yang 
sangat lengkap ini tentu saja memudahkan bagi siapa saja (baik yang sudah
mahir atau pun masih pemula) untuk mengatasi permasalahan yang dihadapi
ketika bekerja dengan \verb=Emacs=, tanpa harus melakukan jelajah ke 
dunia maya.

\section{Buffers}
\verb=Emacs= memiliki 2 jenis \emph{buffer}, yang terkait dengan berkas dan
yang tidak terkait dengan berkas.

\begin{verbatim}
|----------+------------------------------------------------|
| Perintah | Keterangan                                     |
|----------+------------------------------------------------|
| C-x C-f  | Membuka buffer berkas                          |
| C-x C-c  | Menutup emacs                                  |
| C-x k    | Menutup (kill) berkas (buffer)                 |
| C-x C-s  | Menyimpan berkas                               |
| C-x C-w  | Menyimpan buffer sebagai berkas baru (save as) |
|----------+------------------------------------------------|
\end{verbatim}

Berikut ini beberapa perintah untuk bekerja dengan \emph{buffer} pada 
\verb=emacs=:

\begin{verbatim}
|-------------------+--------------------------------------------------|
| Perintah          | Keterangan                                       |
|-------------------+--------------------------------------------------|
| C-x <Panah Kanan> | Berpindah ke buffer berikutnya                   |
| C-x <Panah Kiri>  | Berpindah ke buffer sebelumnya                   |
| C-x b             | Berpindah ke named buffer                        |
| C-x C-b           | Membuka named buffer dalam mode horizontal split |
| C-x C-b           | Daftar buffer yang aktif                         |
| C-x 1             | Kembali ke buffer aktif                          |
| C-x s             | Menyimpan perubahan di semua buffer              |
|-------------------+--------------------------------------------------|
\end{verbatim}

Ada dua perintah untuk menyimpan berkas, yang pertama menggunakan
\verb=C-x C-s=, di mana perintah ini akan menyimpan perubahan hanya pada 
\emph{buffer} yang aktif, dan perintah kedua adalah \verb=C-x s=, yang mana 
akan menyimpan semua perubahan yang ada pada semua \emph{buffer} berkas
yang sedang kita sunting. Perintah kedua ini akan diikuti dengan
\emph{confirmation} apakah kita akan menyimpan perubahan tersebut pada setiap 
\emph{buffer}-nya.

\subsection{ido-mode}
Selain menggunakan \emph{default buffer}, \verb=emacs= juga menyediakan beberapa
\emph{mode} untuk bekerja dengan \emph{buffer} secara lebih efisien. Keduanya 
adalah \verb=iswitchb-mode= dan \verb=ido-mode=.

Keduanya memiliki cara kerja yang sama, namun untuk \verb=emacs 22+=, lebih 
disarankan menggunakan \verb=ido-mode=. Pertanyaan selanjutnya apa beda 
antara \verb=ido-mode= dengan sistem \emph{buffer} bawaan?

Salah satu hal yang mendasar adalah dengan menekan \verb=C-x b=, pada 
\verb=ido-mode= Anda dapat melihat semua daftar \emph{buffer} yang tersedia,
sedangkan jika menggunakan \emph{buffer} bawaan, menekan \verb=C-x b=, Anda 
tidak dapat melihat daftar \emph{buffer}.

Berikut perintah untuk \verb=ido-mode=:

\begin{verbatim}
|--------------+------------------------------------------|
| Perintah     | Keterangan                               |
|--------------+------------------------------------------|
| M-x ido-mode | mengaktifkan ido-mode                    |
| C-x b        | melihat daftar buffer                    |
| C-s          | berpindah (rotate) antar buffer forward  |
| C-r          | berpindah (rotate) antar buffer backward |
| C-x k        | kill buffer                              |
|--------------+------------------------------------------|
\end{verbatim}

\verb=ido-mode= (singkatan dari \emph{interactive-do}) juga mendukung 
\emph{auto-completion} dengan mengetikkan beberapa karakter kemudian tekan 
\verb=TAB=. Anda dapat membuat \emph{buffer} baru dengan mengetikkan 
nama \emph{buffer}, kemudian tekan \verb=ENTER= dua kali.

Apabila Anda ingin selalu menggunakan \verb=ido-mode= setiap kali \verb=emacs= 
dijalankan, tambahkan baris berikut pada berkas \verb=~/.emacs= Anda:

\begin{verbatim}
(ido-mode t) ; set ido-mode true
\end{verbatim}

\subsection{transient-mark-mode}
Mulai dari \verb=emacs 23+=, \emph{transient-mark-mode} diaktifkan secara 
\emph{default}. Anda dapat menonaktifkan \emph{mode} ini dengan menekan:

\begin{verbatim}
|----------+---------------------|
| Perintah | Keterangan          |
|----------+---------------------|
| M-x      | transient-mark-mode |
|----------+---------------------|
\end{verbatim}

Apa keuntungan kita men-\emph{disabled mode} ini? Salah satunya adalah untuk
keperluan \emph{bookmarking}, sebagai contoh Anda sekarang berada pada baris
ke-200, kemudian Anda tekan \verb=M-g g 100= untuk lompat ke baris 100. 
Setelah selesai melakukan penyuntingan pada baris ke-100, Anda dapat lompat
kembali ke baris ke-200 dengan hanya menekan \verb=C-x C-x=.

Pertanyaan selanjutnya adalah apakah sebaiknya \emph{transient-mark-mode}
di-\emph{enable} atau di-\emph{disabled}? Itu semua kembali pada Anda, jika
Anda sering melakukan \emph{copying}, \emph{killing}, dan \emph{yanking}, maka
sebaiknya \emph{transient-mark-mode} diaktifkan. Namun jika Anda sering 
lompat-lompat antar baris, sebaiknya di non-aktifkan, tergantung selera ... 

\subsection{*scratch* buffers}
Setiap kali Anda membuka \verb=emacs=, ada beberapa \emph{buffers} yang secara
otomatis dibuka oleh \verb=emacs= (dinamakan juga \verb=emacs= \emph{buffer}).
Kalau Anda membuka \emph{buffer} lagi, maka itu dinamakan sebagai 
\emph{user buffers}.

Salah satu \verb=emacs= \emph{buffer} adalah \verb=*scratch*=, yang sesuai
dengan namanya, ini adalah tempat di mana Anda dapat melakukan corat-coret
terkait dengan apa yang sedang Anda kerjakan. 

Selain itu, penulis sering memanfaatkan \emph{buffer} ini sebagai 
\emph{command-mode}-nya \verb=emacs=. Kenapa? Karena ketika saya sedang di
depan komputer, kadang ada gangguan internal (baca: anak minta ikut coding),
sementara kita tidak ingin pekerjaan teracak-acak olehnya. Dengan menggunakan
\verb=*scratch*= \emph{buffer} ini, kita tidak perlu lagi khawatir pekerjaan
jadi kacau balau karenanya.

Anda pun dapat menjadikan \verb=*scratch*= \emph{buffer} ini sebagai media
untuk belajar bahasa pemrograman \verb=Lisp=, berikut ini contoh operasi
aritmatika sederhana:

\begin{verbatim}
(+ 5 3)
\end{verbatim}

Tekan \verb=C-j=, maka \verb=emacs= akan mengevaluasi baris tersebut dan 
memberikan hasilnya tepat di garis setelahnya.

\begin{verbatim}
(+ 5 3)
8
\end{verbatim}

\section{Bookmarks}
Sesuai dengan namanya, \emph{bookmarks} memungkinkan kita untuk menyimpan
posisi \emph{point} dalam berkas. \emph{Bookmarks} ditandai dengan sebuah nama
yang akan disimpan selama \emph{sessions} \verb=emacs= masih berjalan. Bahkan
ketika isi dari berkas berubah, \emph{bookmark} masih dapat bekerja dengan baik.
Hal ini dikarekanakan \emph{bookmarks} tidak hanya menyimpan posisi \emph{point}
namun juga menyimpan teks apa saja yang ada di sekitar \emph{point}. Namun jika
berkas sudah banyak berubah, mungkin saja \emph{bookmarks} tidak akan bekerja
lagi.

Berikut ini perintah-perintah terkait dengan \emph{bookmarks}:

\begin{verbatim}
|------------------------+-----------------------------------------|
| Perintah               | Keterangan                              |
|------------------------+-----------------------------------------|
| C-x r m BOOKMARK <RET> | menyimpan bookmark dengan nama BOOKMARK |
| C-x r b BOOKMARK <RET> | memanggil bookmark dengan nama BOOKMARK |
| C-x r l                | melihat daftar bookmark                 |
| M-x bookmark-save      | menyimpan bookmark ke dalam berkas      |
| M-x edit-bookmarks     | menyunting bookmark                     |
|------------------------+-----------------------------------------|
\end{verbatim}

\section{Registers}
Fitur \emph{register} dapat menyimpan teks yang Anda salin, untuk kemudian 
Anda tempel di mana pun Anda inginkan (tanpa harus mencari pada \emph{ring}).
Caranya adalah dengan menyimpan teks dalam \emph{register} yang dapat Anda
panggil setiap Anda butuhkan. 

Berikut cara untuk bekerja dengan \emph{registers}

\begin{verbatim}
|---------------------------------+--------------------------------------|
| Perintah                        | Keterangan                           |
|---------------------------------+--------------------------------------|
| C-x r s k                       | copy region ke dalam register r      |
|                                 | (copy-to-register)                   |
| C-x r i k                       | panggil register k (insert-register) |
| M-x append-to-register <RET> k  | tambahkan (append) region ke dalam   |
|                                 | register k                           |
| M-x prepend-to-register <RET> k | prepend region ke dalam register k   |
|---------------------------------+--------------------------------------|
\end{verbatim}

\section{Search and Replace}
Salah satu fungsi penting sebuah program penyunting teks adalah fungsi cari
dan ganti (\emph{search and replace}). Pada bagian ini kita akan mempelajari
bagaimana \verb=emacs= melakukan fungsi tersebut.

Pencarian pada \verb=emacs= dapat dibagi menjadi dua, pencarian maju dan 
pencarian mundur. Pencarian maju lakukan dengan menekan:

\begin{verbatim}
|----------+-----------------------------------|
| Perintah | Keterangan                        |
|----------+-----------------------------------|
| C-s      | Mengaktifkan mode search maju     |
| C-s      | Mencari next occurence (jika ada) |
| Enter    | Keluar dari search mode           |
|----------+-----------------------------------|
\end{verbatim}

Pencarian mundur:

\begin{verbatim}
|----------+---------------------------------------|
| Perintah | Keterangan                            |
|----------+---------------------------------------|
| C-r      | Mengaktifkan mode search mundur       |
| C-r      | Mencari previous occurence (jika ada) |
| Enter    | Keluar dari search mode               |
|----------+---------------------------------------|
\end{verbatim}

Untuk mencari \emph{previous occurence} pada pencarian mundur dapat juga
dengan menekan tombol \verb=Backspace=.

Anda dapat juga mengubah mode pencarian maju ke mundur (atau sebaliknya)
dengan menekan \verb=C-s= atau \verb=C-r=.

\subsection{Replace String}

Cara pertama adalah menggunakan \emph{eXtended Command} yang dapat diakses
menggunakan \verb=M-x=. Sebagai contoh kita memiliki kalimat berikut:

\begin{verbatim}
Ini ibu budi
\end{verbatim}

Untuk mengganti kata \verb=ibu= menjadi \verb=bapak=, berikut perintah yang
harus dilakukan:

\begin{verbatim}
|--------------------------------------------------+--------------------------|
| Perintah                                         | Keterangan               |
|--------------------------------------------------+--------------------------|
| C-a                                              | Menuju ke awal baris     |
| C-<SPC>                                          | Mark Set                 |
| C-e                                              | Sorot sampai akhir baris |
| M-x repl<Tab>s<Tab><Enter>ibu<Enter>bapak<Enter> |                          |
|--------------------------------------------------+--------------------------|
\end{verbatim}

Cara kedua adalah dengan menggunakan perintah \verb=replace-string=.
Caranya adalah dengan menyorot daerah yang mengandung kata yang akan diganti
kemudian tekan \verb=M-x replace-string= tekan \verb=enter=, masukkan kata
yang ingin diganti, tekan \verb=enter= lagi, kemudian masukkan kata yang
pengganti.

\begin{verbatim}
|---------------------------+------------------------------------|
| Perintah                  | Keterangan                         |
|---------------------------+------------------------------------|
| C-space                   | Mulai menyorot                     |
| Motion                    | Tandai daerah yang mengandung kata |
| M-x replace-string        |                                    |
| Enter                     |                                    |
| Masukkan kata yang dicari |                                    |
| Enter                     |                                    |
| Masukkan kata pengganti   |                                    |
| Enter                     |                                    |
|---------------------------+------------------------------------|
\end{verbatim}

\subsection{Query Replace}
Fitur ini akan mengganti setiap perulangan kata yang dicari pada seluruh dokumen
namun dengan proses konfirmasi sebelumnya. Berikut ini perintahnya:

\begin{verbatim}
|----------------------------------------------------------------+------------|
| Perintah                                                       | Keterangan |
|----------------------------------------------------------------+------------|
| M-x query-replace <teks awal> <Enter> <teks pengganti> <Enter> |            |
| atau                                                           |            |
| M-% <teks awal> <Enter> <teks pengganti> <Enter>               |            |
|----------------------------------------------------------------+------------|
\end{verbatim}

Misal:

\begin{verbatim}
Saya seorang programmer
\end{verbatim}

Anda ingin mengganti kata \verb=programmer= menjadi \verb=coder=:

\begin{verbatim}
|----------------------------------------------------------------|
| Perintah                                                       |
|----------------------------------------------------------------|
| C-a                                                            |
| M-x query-replace <Enter> <programmer> <Enter> <coder> <Enter> |
|----------------------------------------------------------------|
\end{verbatim}

\section{Modes}
\verb=emacs= memiliki banyak \emph{mode}, berikut ini beberapa \emph{mode} yang sering
penulis pakai:

\subsection{Dired-Mode}
Untuk melakukan pekerjaan manipulasi berkas atau direktori, \verb=emacs=
menyediakan sebuah \emph{mode} bernama \verb=Dired mode=. Sesuai dengan namanya,
\emph{Dired} (\emph{directory editor}) mempermudah kita dalam melakukan manipulasi berkas 
atau direktori dengan mudahnya.

Untuk berpindah ke \verb=dired mode=, kita dapat mengetik:

\begin{verbatim}
|-----------+---------------------------|
| Perintah  | Keterangan                |
|-----------+---------------------------|
| C-x d     | aktifkan dired mode, atau |
| M-x dired | aktifkan dired mode       |
|-----------+---------------------------|
\end{verbatim}

Dari sini, Anda dapat melakukan manipulasi berkas dan direktori, seperti 
misalnya untuk menghapus sebuah berkas, tekan \verb=D= dan seterusnya ...

Berikut beberapa perintah terkait dengan \emph{dired mode}

\begin{verbatim}
|----------+-----------------------------------------------------------|
| Perintah | Keterangan                                                |
|----------+-----------------------------------------------------------|
| <RETURN> | membuka berkas                                            |
| q        | menutup direktori                                         |
| C        | menyalin berkas                                           |
| R        | mengubah (move) nama berkas                               |
| D        | menghapus berkas                                          |
| +        | membuat direktori baru                                    |
| Z        | compress / decompress berkas menggunakan gzip             |
| m        | memberi tanda pada berkas                                 |
| u        | menghilangkan tanda pada berkas                           |
| U        | menghilangkan tanda pada semua berkas yang sudah ditandai |
| % m      | memberi tanda menggunakan regex                           |
| g        | refresh direktori                                         |
| ^        | kembali ke direktori atasnya                              |
| w        | menyalin nama berkas                                      |
| 0w       | menyalin full path dan nama berkas                        |
|----------+-----------------------------------------------------------|
\end{verbatim}

Mungkin ada yang bingung bagaimana cara memindahkan berkas / direktori
pada \emph{dired-mode}? Gampang kok tinggal pilih satu [beberapa] atau 
direktori yang ingin Anda pindahkan, kemudian tekan \verb=R=.

\subsection{org-mode}
Apa itu \emph{org-mode}? 

\emph{Org mode is for keeping notes, maintaining TODO lists, planning projects, and authoring documents with a fast and effective plain-text system.}\footnote{http://www.orgmode.org}

\emph{org-mode} merupakan salah satu fitur andalan \verb=emacs=. Seperti apakah
\emph{org-mode} itu? Mari kita langsung ke contoh saja ...

\subsubsection{Instalasi}

Pertama kali mari kita pasang \verb=org-mode=, pada \verb=Ubuntu=, ketik:

\begin{verbatim}
sudo apt-get install org-mode
\end{verbatim}

\subsubsection{Aktifkan org-mode}

Buka berkas baru, kemudian aktifkan \emph{org-mode} dengan:

\begin{verbatim}
|--------------+-------------------|
| Perintah     | Keterangan        |
|--------------+-------------------|
| M-x org-mode | Aktifkan org-mode |
|--------------+-------------------|
\end{verbatim}

Jika Anda ingin \emph{org-mode} terpanggil secara otomatis setiap kali
\verb=emacs= dinyalakan, silakan sunting berkas \verb=.emacs= Anda
dan isikan baris berikut:

\begin{verbatim}
;; Must have org-mode loaded before we can configure org-babel
(require 'org-install)
\end{verbatim}

\subsubsection{Navigasi Cepat}
Untuk berpindah secara cepat antar headline, berikut tombol kombinasinya:

\begin{verbatim}
|----------+----------------------------------|
| Perintah | Keterangan                       |
|----------+----------------------------------|
| C-c C-n  | Berpindah ke headline berikutnya |
| C-c C-p  | Berpindah ke headline sebelumnya |
|----------+----------------------------------|
\end{verbatim}

\subsubsection{Format Standar Dokumen Org}

Format dasar dokumen \verb=org= adalah:

\begin{verbatim}
|------+------------------------|
| Kode | Keterangan             |
|------+------------------------|
| *    | bagian utama           |
| **   | subbagian tingkat dua  |
| ***  | subbagian tingkat tiga |
| -    | list                   |
| /i/  | cetak miring           |
| *b/  | cetak tebal            |
| _u_  | garis bawah            |
| =c=  | true type font (code)  |
|------+------------------------|
\end{verbatim}

Untuk mengetahui format lain, tekan \texttt{C-c C-x C-f}.

\subsubsection{Promote / Demote Headline}
Fitur ini berguna untuk mengubah secara cepat sebuah \emph{headline} yang
tadinya bintang satu, menjadi bintang dua dan seterusnya.

\begin{verbatim}
|------------------------+---------------------------|
| Perintah               | Keterangan                |
|------------------------+---------------------------|
| M-S-Panah Kanan / Kiri | Promote / Demote Headline |
|------------------------+---------------------------|
\end{verbatim}

\subsubsection{Ekspor Dokumen}
Dengan menggunakan \emph{org-mode}, Anda dapat dengan mudah meng-\emph{export}
dokumen Anda ke berbagai format, misal \verb=HTML=, \verb=LaTex= maupun 
format-format lain.

Berikut contoh berkas yang menggunakan \emph{org-mode}

\begin{verbatim}
* Judul
  Isi dari judul
** Sub Judul 1
*** Sub Judul 2
\end{verbatim}

Mari kita \emph{export} dokumen di atas ke dalam format \verb=HTML= dengan cara:

\begin{verbatim}
|----------+------------------------------|
| Perintah | Keterangan                   |
|----------+------------------------------|
| C-c C-e  | Aktifkan export mode         |
| b        | Pilih b untuk export ke html |
| <RETURN> | Tekan Enter                  |
|----------+------------------------------|
\end{verbatim}

Silakan ditunggu sebentar, berikutnya \verb=emacs= akan menampilkan hasil
\emph{export} tadi dalam aplikasi \emph{browser}.

Perlu diketahui bahwa \emph{option} \verb=b= di atas hanyalah salah satu dari
sekian banyak pilihan format dokumen yang akan di-\emph{export}. Untuk 
format-format lain, silakan di-\emph{explore} lebih lanjut.

\subsubsection{Expand and Collapse Tree}
Jika Anda sudah pernah menggunakan konsep \emph{folding}, tentu tidak akan
bingung, \emph{org-mode} mendukung hal tersebut dengan cara menekan:

\begin{verbatim}
|----------+------------------------|
| Perintah | Keterangan             |
|----------+------------------------|
| <TAB>    | rotate current subtree |
| atau     |                        |
| S-<TAB>  | rotate entire buffer   |
|----------+------------------------|
\end{verbatim}

\verb=S= adalah tombol \verb=Shift=.

\subsubsection{Format Font}
Untuk memformat huruf dalam lingkungan \emph{org-mode}, cukup tambahkan
karakter-karakter khusus berikut di awal dan di akhir kata yang ingin 
Anda format.

Misalnya, untuk mencetak huruf tebal, tambahkan tanda \verb=*= di awal
dan di akhir kata, \verb=*ini akan dicetak tebal*=.

Berikut daftar lengkap format huruf pada \emph{org-mode}:

\begin{verbatim}
|-------------+-------------------------|
| Perintah    | Keterangan              |
|-------------+-------------------------|
| *bold*      | cetak tebal             |
| /italics/   | cetak miring            |
| _underline_ | garis bawah             |
| =code=      | format untuk kode       |
| ~verbatim~  | format font mesin ketik |
|-------------+-------------------------|
\end{verbatim}

\subsubsection{Comment}
Untuk melakukan \emph{comment}, gunakan tanda \verb=#= di awal baris, atau
tambahkan kata \verb=COMMENT= jika Anda ingin memberi \emph{comment} pada
sebuah \emph{entry}, sebagai contoh:

\begin{verbatim}
# Ini tidak akan di export
** COMMENT subbagian ini juga tidak akan di export
isi dari subbagian ini juga tidak akan di export
\end{verbatim}

\subsubsection{Lists}
Berikut ini adalah cara membuat \emph{lists} pada \emph{org-mode}:

\begin{verbatim}
- TODO
  - Mandi
  - Sarapan
  - Coding
\end{verbatim}

\noindent\textbf{Merubah Jenis List}

Ada beberapa macam list dalam \emph{org-mode}, seperti angka, tanda kurang,
angka dengan kurung, tanda tambah, dsb. Anda dapat merubahnya secara 
otomatis menggunakan tombol:

\begin{verbatim}
|----------------------+------------------------------|
| Perintah             | Keterangan                   |
|----------------------+------------------------------|
| S-Panah Kanan / Kiri | Merubah jenis list numbering |
|----------------------+------------------------------|
\end{verbatim}

\subsubsection{Partial Export}
Selain meng-\emph{export} seluruh dokumen, Anda pun dapat melakukan 
\emph{export} hanya pada bagian tertentu, dengan cara menyorot bagian yang 
ingin Anda sorot, kemudian tekan \verb=C-c C-e b= untuk \emph{export} ke 
format \verb=HTML=.

\subsubsection{Todo Items}
Untuk membuat daftar \emph{todo items}, lihat contoh berikut:

\begin{verbatim}
- [X] emacs
- [ ] vim
- [ ] others
\end{verbatim}

Untuk memberi (menghilangkan) centang, gunakan perintah:

\begin{verbatim}
|----------+---------------------------------------|
| Perintah | Keterangan                            |
|----------+---------------------------------------|
| C-c C-c  | Memberi / menghilangkan tanda centang |
|----------+---------------------------------------|
\end{verbatim}

\subsubsection{TODO and DONE items}
Selain menggunakan sistem \emph{checklist} di atas, Anda pun dapat membuat
\emph{TODO DONE} menggunakan format sebagai berikut:

\begin{verbatim}
* DONE Go to school
* TODO Doing homework
\end{verbatim}

Berikut cara membuatnya:

\begin{verbatim}
* <SPACE> C-c C-t <Go to school>
\end{verbatim}

Atau:

\begin{verbatim}
* S-<RightArrow>
\end{verbatim}

Untuk mengubah dari \emph{TODO} menjadi \emph{DONE}, cukup tekan
\verb=C-c C-t= sekali lagi.

\subsubsection{Date and Time}
Untuk memasukkan \emph{date and time stamp}, gunakan perintah berikut:

\begin{verbatim}
|-----------+---------------------------------|
| Perintah  | Keterangan                      |
|-----------+---------------------------------|
| C-c .     | memasukkan timestampe           |
| C-u C-c . | memasukkan timestamp dengan jam |
|-----------+---------------------------------|
\end{verbatim}

Begitu Anda tekan \verb=RETURN=, maka akan muncul kalendar di bagian bawah
layar \verb=emacs= Anda. Selanjutnya Anda dapat melakukan navigasi dengan
menggunakan \emph{mouse} Anda.

\begin{verbatim}
<2014-05-15 Thu>
<2013-01-14 Mon 08:57>
\end{verbatim}

Anda dapat menyunting tanggal di atas dengan mengarahkan \emph{point} Anda
ke angka tahun, kemudian tekan \verb=S-Up= atau \verb=S-Down= untuk 
menambah atau mengurangi angka tahun, begitu pula dengan bulan, tanggal, hari
dapat Anda ubah seperti cara di atas.

Khusus untuk mengganti hari (tanggal), Anda dapat menggunakan tombol:

\begin{verbatim}
|--------------+---------------------|
| Perintah     | Keterangan          |
|--------------+---------------------|
| S-LeftArrow  | previous date / day |
| S-RightArrow | next date / day     |
|--------------+---------------------|
\end{verbatim}

\subsubsection{Links}
Berikut ini format untuk membuat \emph{links} dalam \emph{org-mode}:

\begin{verbatim}
http://www.infotiket.com
\end{verbatim}

atau:

\begin{verbatim}
[[http://www.elapak.com][Situs Jual Beli Indonesia]]
\end{verbatim}

Begitu Anda tutup tanda \emph{square bracket} yang terakhir, format \emph{URL}
di atas otomatis akan berubah menjadi sebuah \emph{link} dengan \emph{anchor}:

\begin{verbatim}
Situs Jual Beli Indonesia
\end{verbatim}

Untuk menyunting \emph{link} di atas, tempatkan \emph{point} pada \emph{URL},
kemudian tekan:

\begin{verbatim}
|----------+--------------------|
| Perintah | Keterangan         |
|----------+--------------------|
| C-c C-l  | Sunting link (url) |
|----------+--------------------|
\end{verbatim}

\subsubsection{Tagging}
Berikut cara membuat \emph{tagging} dalam \verb=emacs=:

Arahkan \emph{mouse} ke salah satu bagian atau subbagian, kemudian tekan:

\begin{verbatim}
|----------+---------------------|
| Perintah | Keterangan          |
|----------+---------------------|
| C-c C-c  | Menambahkan tagging |
|----------+---------------------|
\end{verbatim}

Nanti akan muncul \emph{prompt} di bagian bawah layar \verb=emacs=, selanjutnya
isikan \emph{tags} yang Anda inginkan. Untuk menyunting \emph{tags}, tinggal 
ulangi perintah \verb=C-c C-c= di atas.

\subsubsection{Membuat Tabel}

Berikut cara membuat tabel dalam \emph{org-mode}:

\begin{verbatim}
| kolom 1 | kolom 2 |
|---<TAB>
some data<TAB>123
<ENTER>
<ENTER>
<ENTER>
<TAB>
<TAB>
\end{verbatim}

Dan berikut hasilnya:

\begin{verbatim}
| kolom 1   | kolom 2 |
|-----------+---------|
| some data |     123 |
|           |         |
|           |         |
|           |         |
|           |         |
|           |         |
\end{verbatim}

Anda dapat melakukan operasi matematika dengan tabel tersebut, silakan
di-\emph{explore} sendiri.

\subsubsection{Image}
Untuk menampilkan gambar di org-mode yang latex friendly, gunakan kode berikut:

\begin{verbatim}
#+CAPTION: Plot TLKM.JK dan SMA20
#+NAME: fig:tlkm
#+ATTR_LaTeX: placement=[H]
[[./images/tlkm.png]]
\end{verbatim}

\subsubsection{Source Code}
Sebelum kita dapat menjalankan kode dalam \emph{org-mode}, pastikan kita
sudah melakukan pengaturan \emph{org-babel} pada berkas \verb=~/.emacs=.
Isi berkas tersebut dengan baris berikut:

\begin{verbatim}
; Must have org-mode loaded before we can configure org-babel
(require 'org-install)

;Some initial languages we want org-babel to support
(org-babel-do-load-languages
 'org-babel-load-languages
 '(
   (sh . t)
   (python . t)
   (R . t)
   (ruby . t)
   (ditaa . t)
   (dot . t)
   (octave . t)
   (sqlite . t)
   (perl . t)
   ))
\end{verbatim}

Selanjutnya silakan \emph{restart} \verb=emacs=.

Untuk menuliskan \emph{source code}, kita dapat menggunakan \emph{tag} berikut:

\begin{verbatim}
#+BEGIN_SRC sh :exports both
echo "Hello World!" # C-c C-c
#+END_SRC
\end{verbatim}

Tempatkan \emph{point} pada kode, kemudian tekan \verb=C-c C-c=, dan hasilnya:

\begin{verbatim}
#+BEGIN_SRC sh :exports both
echo "Hello World!" # C-c C-c
#+END_SRC

#+results:
: Hello World!
\end{verbatim}

Contoh lain untuk kode \verb=python=:

\begin{verbatim}
#+BEGIN_SRC python :exports both
return "Hello Python!" # C-c C-c
#+END_SRC

#+results:
: Hello Python!
\end{verbatim}

Mungkin bagi Anda yang terbiasa dengan bahasa pemrograman \verb=python= 
bertanya-tanya, kenapa menggunakan \verb=return=, dan tidak menggunakan
\verb=print=. Saya sendiri sampai saat ini belum apa alasan \verb=emacs=
menggunakan \emph{keyword} \verb=return=, yang jelas jika Anda coba 
mengganti dengan \verb=print=, maka \emph{result}-nya menjadi
\verb=none=.

\emph{Note}: \emph{org-babel} ini hanya akan jalan jika Anda sudah 
memasang paket \verb=texlive-latex-extra= pada sistem operasi Anda. Jika
Anda pengguna Ubuntu, tentunya gunakan:

\begin{verbatim}
sudo apt-get install texlive-latex-extra
\end{verbatim}

\subsubsection{Paket fancyhdr}
Org mode mendukung paket fancyhdr seperti ketika membuat dokumen Latex
pada umumnya, berikut caranya:

\begin{verbatim}
#+LATEX_HEADER: \usepackage{fancyhdr}
#+LATEX_HEADER: \pagestyle{fancy}
#+LATEX_HEADER: \cfoot{}
#+LATEX_HEADER: \rfoot{\thepage}
\end{verbatim}

\subsubsection{Syntax Highlighting}
Apabila Anda ingin fitur /syntax highlighting/ aktif pada berkas PDF,
caranya adalah dengan mengaktifkan paket \verb=minted=.

Pertama, pasang pustaka pygments, dengan cara:

\begin{verbatim}
$ sudo pip install pygments
\end{verbatim}

Setelah itu, sunting berkas \verb=.emacs= Anda:

\begin{verbatim}
(require 'org-latex)
(setq org-export-latex-listings 'minted)
(add-to-list 'org-export-latex-packages-alist '("" "minted"))
\end{verbatim}

Terakhir, masukkan kode berikut di \emph{header} berkas \verb=Org= Anda:
\begin{verbatim}
#+LaTeX_HEADER: \usepackage{minted}
\end{verbatim}

\emph{Voila}, semua kode sekarang menjadi berwarna!

\subsubsection{Mengganti ukuran font pada table}
Berikut ini contoh mengganti ukuran font tabel menjadi footnotesize
\begin{verbatim}
#+LATEX: {\footnotesize
#+ATTR_LaTeX: align=cl placement=[H]
|----+--------------------------------------------------------|
| H0 | Korelasi antara harga saham tinggi sehingga harga      |
|    | saham yang akan datang dapat diramalkan dengan         |
|    | menggunakan harga di masa lalu                         |
|----+--------------------------------------------------------|
| H1 | Korelasi antara harga saham rendah sehingga harga      |
|    | saham di masa depan tidak dapat diprediksi menggunakan |
|    | data harga masa lalu                                   |
|----+--------------------------------------------------------|
\end{verbatim}

\subsubsection{Contoh Berkas Org-Mode untuk Catatan Harian}
Berikut ini contoh penggunaan berkas \emph{org-mode} untuk \emph{daily log}:

\begin{verbatim}
* Jan 14 in Yogyakarta <2013-01-14 Mon>			:day:

** Writing emacs documentation

   Today we write about org-mode / org-babel
\end{verbatim}

Salah satu hal yang membuat saya suka dengan \verb=emacs= adalah komunitas
pengguna \verb=emacs= sangat mendorong kita sebagai pengguna \verb=emacs=
untuk membuat catatan harian seperti di atas, hal ini berlaku tidak hanya
bagi peneliti namun juga \emph{programmer} pengguna \verb=emacs=.

\subsection{DocView-Mode}
Tahukah Anda, bahwa Anda dapat membuka berkas \verb=PDF= di \verb=emacs=?
\verb=Emacs= menyediakan \emph{mode} khusus untuk ini, yakni 
\emph{docview-mode}.

Caranya sama dengan ketika Anda membuka berkas baru, yakni dengan menekan
\verb=C-x C-f= kemudian pilih berkas \verb=PDF= Anda, kemudian tekan 
\verb=Enter=.

Berikut ini adalah perintah untuk berinteraksi ketika bekerja dengan 
\emph{DocView-mode}:

\begin{verbatim}
|--------------------------------+-----------------------------------|
| Perintah                       | Keterangan                        |
|--------------------------------+-----------------------------------|
| <SPC>                          | Menuju ke halaman berikutnya      |
| <Backspace>                    | Menuju ke halaman sebelumnya      |
| +                              | Memperbesar halaman (Zoom in)     |
| -                              | Memperkecil halaman (Zoom out)    |
| M-<                            | Menuju ke awal dokumen            |
| M->                            | Menuju ke akhir dokumen           |
| M-g g <num> atau M-g M-g <num> | Menuju ke halaman tertentu        |
| C-s                            | forward search                    |
| C-r                            | backward search                   |
| C-u C-s                        | memulai pencarian baru (forward)  |
| C-u C-r                        | memulai pencarian baru (backward) |
| s s                            | slicing halaman                   |
| s m                            | slicing dengan mouse              |
| s r                            | reset ke default                  |
| q                              | keluar dari mode docview          |
|--------------------------------+-----------------------------------|
\end{verbatim}

Terkait dengan fitur \emph{search}, \emph{docview-mode} memiliki beberapa
perintah khusus, misalnya ketika Anda sudah selesai melakukan pencarian,
kemudian ingin melanjutkan ke pencarian berikutnya, maka Anda harus menekan
\verb=C-u C-s= atau \verb=C-u C-r=. Setelah menekan tombol \verb=Enter=, 
maka hasil pencarian pada dokumen tidak akan di-\emph{highlight}, melainkan
akan ditampilkan dalam bentuk \emph{tooltip}, yang akan muncul ketika Anda
melakukan \emph{hover} perangkat \emph{mouse} Anda pada halaman tersebut.

Berkas lain yang didukung oleh \emph{docview-mode} ini adalah \verb=DVI=, 
\verb=PostScript(PS)=, \verb=OpenDocument=, dan 
\verb=Microsoft Office documents=.

\subsection{viper-mode}
Bagi Anda yang suka dengan model navigasi \verb=Vim=, Anda dapat
mengaktifkan \emph{viper-mode}. \emph{Mode} ini mendukung semua perintah
navigasi seperti seperti ketika bekerja dengan \verb=Vim=, misal \verb=w=
untuk menggerakkan \emph{point} ke kata selanjutnya, \verb=G= untuk bergerak
ke akhir \emph{buffer}, dan perintah-perintah navigasi lainnya.

Apabila Anda tertarik dengan \verb=Vim=, penulis juga menyediakan tutorial
\verb=Vim= dalam format \verb=PDF=, dan bisa diunduh di alamat berikut:

\begin{verbatim}
http://dl.dropbox.com/u/5052616/vim_docs.pdf
\end{verbatim}

\section{Packages dan Repo}
Seperti halnya sistem operasi, emacs juga memiliki server repo yang 
berisi paket yang dapat kita tambahkan. Paket ini mirip addons atau 
plugin. By default, repo emacs dapat kita akses dengan:

\begin{verbatim}
M-x package-list-packages
\end{verbatim}

Perhatikan bahwa command ini membutuhkan akses internet dan 
berikut tampilannya:

\vspace{12pt}

\includegraphics[scale=0.6]{images/repo1.png} 

\vspace{12pt}

kita dapat melakukan search dengan \verb=C-s= seperti biasa. Misal 
kita ingin mencari paket \verb=markdown=:

\begin{verbatim}
C-s markdown
\end{verbatim}

\vspace{12pt}

\includegraphics[scale=0.6]{images/repo_search.png} 

\vspace{12pt}

Perhatikan bahwa paket markdown sudah terpasang di emacs saya. 
Pada sesi berikutnya kita akan belajar menambahkan paket di emacs.

\subsection{Menambahkan/Install Packages}
\subsection{Menambahkan Repo Alternatif}

\section{Menjalankan Shell dari emacs}
\verb=Emacs= memiliki berbagai macam perintah untuk berinteraksi dengan
\emph{shell}:

\begin{verbatim}
|-------------------+--------------------------------------------------|
| Perintah          | Keterangan                                       |
|-------------------+--------------------------------------------------|
| M-! cmd <RET>     | menjalankan perintah shell dan menampilkan hasil |
| M-/ cmd <RET>     | menjalankan perintah shell dengan region sebagai |
|                   | input.                                           |
| M-& cmd <RET>     | menjalankan perintah shell asynchronously, dan   |
|                   | menampilkan hasil.                               |
| M-x shell         | menjalankan subshell dengan input dan output     |
|                   | melalui buffer emacs. Beberapa command tidak     |
|                   | dapat berjalan pada mode ini.                    |
| C-u M-x <RET>     | Menjalankan shell kedua (multiple shell).        |
| M-x term          | menjalankan subshell dengan input dan output     |
|                   | melalui buffer emacs. Mode ini mendukung semua   |
|                   | command seperti kita membuka aplikasi terminal.  |
| M-x rename-buffer | Mengganti nama buffer.                           |
|-------------------+--------------------------------------------------|
\end{verbatim}

\verb=Shell= tersebut tidak ubahnya seperti \emph{buffer}, sehingga Anda dapat 
memanggilnya (menutupnya) seperti ketika kita bekerja dengan \emph{buffer}.

Untuk mengulangi perintah sebelumnya pada \verb=shell=, tekan:

\begin{verbatim}
C-<Panah atas>
\end{verbatim}

Penulis sendiri tidak mengetahui kenapa \emph{shell mode} tidak mendukung
semua perintah yang ada pada \verb=Shell=, misal perintah \verb=mkvirtualenv=
yang sering penulis pakai ketika bekerja dengan \verb=Python= tidak dapat 
bekerja pada \emph{shell mode}, namun dapat bekerja pada \emph{term mode}.

Anda dapat menjalankan lebih dari satu \emph{shell mode} dengan cara 
menekan \verb=C-u M-x shell= diikuti dengan nama \emph{shell} yang Anda 
inginkan, atau cukup tekan \verb=Enter= untuk memberi nama secara 
\emph{default} (\verb=*shell*<2>=). Untuk mengganti nama \emph{buffer}
Anda dapat menggunakan perintah \verb=M-x rename-buffer=.

Khusus untuk mode \emph{term}, jangan bingung ketika menggunakannya, ada
beberapa perbedaan perintah, semua perintah \verb=emacs= yang diawali dengan
\verb=C-x= berubah menjadi \verb=C-c=, jadi semisal Anda ingin berpindah
\emph{buffer}, tekan \verb=C-c b=. Penulis sendiri sebisa mungkin menghindari
mode \emph{term}, karena terkadang lupa harus mengganti perintah \verb=C-x=
menjadi \verb=C-c=, namun untuk hal-hal yang memang harus dijalankan melalui
\verb=terminal=, mau tidak mau penulis menggunakan mode \emph{term}.

\section{Tips dan Tricks}

\subsection{Multiple Windows}
Untuk meningkatkan efisiensi dalam bekerja, Anda dapat menggunakan fitur
\emph{multiple windows}.

\begin{verbatim}
|---------------+---------------------------------------------------------|
| Perintah      | Keterangan                                              |
|---------------+---------------------------------------------------------|
| C-x 3         | Split Vertical                                          |
| C-x 2         | Split Horizontal                                        |
| C-x 1         | Menutup semua window kecuali window di mana point aktif |
| C-x 0         | Menutup window aktif (tempat point berada)              |
| C-x o         | Berpindah antar window                                  |
| C-M-v         | Menggulung layar window dibawahnya                      |
| C-x 4 C-f     | Membuka berkas (buffer) pada window sebelahnya          |
| C-u 6 C-x ^   | Memperbesar ukuran window sebanyak 6 baris              |
| C-u - 6 C-x ^ | Memperkecil ukuran window sebanyak 6 baris              |
|---------------+---------------------------------------------------------|
\end{verbatim}

Perintah \verb=C-M-v= berguna ketika kita ingin menjadikan \emph{window} lain
sebagai referensi terhadap pekerjaan kita, tanpa harus berpindah \emph{window},
Anda dapat menggulung layar referensi tersebut, dan \emph{point} tetap berada 
pada dokumen (berkas) yang sedang kita kerjakan.

Perintah \verb=C-x 4 C-f= berguna untuk membuka \emph{buffer} berkas pada
\emph{window} sebelahnya.

\subsection{Multiple Frame}
Apabila Anda bekerja dengan \emph{GUI}, maka membuat \emph{frame} baru adalah
seperti ketika Anda menjalankan \verb=emacs= pertama kali. Fitur ini tidak 
berguna ketika Anda bekerja dengan \verb=emacs= dalam \emph{mode} 
\verb=terminal= (\verb=emacs -nw=).

Berikut ini beberapa perintah untuk bekerja dengan \emph{multi-frame}:

\begin{verbatim}
|---------------------+----------------------------------------------|
| Perintah            | Keterangan                                   |
|---------------------+----------------------------------------------|
| C-x 5 f <some_file> | Membuka berkas dalam frame baru              |
| C-x 5 o             | Berpindah antar frame                        |
| C-x 5 b             | Berpindah ke buffer yang ada pada frame lain |
| C-x 5 0             | Menutup frame                                |
|---------------------+----------------------------------------------|
\end{verbatim}

Jika diperhatikan perintah-perintah di atas sama dengan perintah ketika kita
bekerja dalam \emph{single-frame}, yang membedakan adalah kita menambahkan
angka \verb=5= pada setiap perintah yang kita lakukan untuk bekerja dengan 
\emph{multi-frame}.

\subsection{Menggagalkan Perintah}
Jika Anda membuka \verb=emacs= pertama kali, maka pada baris ke-13 Anda akan
temukan kalimat berikut:

\begin{verbatim}
To quit a partially entered command, type Control-g.
\end{verbatim}

\begin{verbatim}
|----------+-----------------------|
| Perintah | Keterangan            |
|----------+-----------------------|
| C-g      | Menggagalkan perintah |
|----------+-----------------------|
\end{verbatim}

Biar lebih jelas, silakan perhatikan contoh berikut:

Anda ingin membuka berkas dengan menekan tombol \verb=C-x C-f=, namun 
tiba-tiba Anda ingin menggagalkan perintah tersebut, caranya cukup hanya
dengan menekan \verb=C-g=, dengan begitu Anda dapat memberikan perintah
baru ke \verb=emacs=.

\subsection{Memunculkan Menu Bantuan (Help)}
Anda dapat menampilkan menu bantuan (\emph{help}) dengan menekan tombol
berikut:

\begin{verbatim}
|----------------------------+-----------------------------------------|
| Perintah                   | Keterangan                              |
|----------------------------+-----------------------------------------|
| F1                         |                                         |
| M-x help <RET>             |                                         |
| C-h <Kombinasi Perintah>   |                                         |
| C-h ?                      |                                         |
| C-h c <Kombinasi Perintah> | help singkat, misal: C-h c C-a          |
| C-h k <Kombinasi Perintah> | help panjang, misal: C-h k C-a          |
| C-h f <Deskripsi Fungsi>   | help berdasar fungsi,                   |
|                            | misal: C-h previous line                |
| C-h m                      | help untuk mode yang sedang digunakan   |
| C-h v                      | dokumentasi variabel                    |
| C-h a <keyword>            | help berdasar keyword, misal C-h a file |
| C-h i                      | emacs manual, a.k.a Info                |
|----------------------------+-----------------------------------------|
\end{verbatim}

Sebagai contoh, Anda ingin mengetahui untuk apakah kombinasi perintah
\verb=M-a=:

\begin{verbatim}
C-h k M-a
\end{verbatim}

Maka \verb=emacs= akan memunculkan \emph{window} baru dalam \emph{mode split}
horisontal. Untuk kembali melakukan penyuntingan, cukup tutup \emph{window}
tersebut dengan perintah yang sama dengan ketika kita ingin menutup 
\emph{split window}, yakni:

\begin{verbatim}
|----------+------------------------------------------------|
| Perintah | Keterangan                                     |
|----------+------------------------------------------------|
| C-x 1    | Hanya menampilkan 1 window dimana point berada |
|----------+------------------------------------------------|
\end{verbatim}

Perintah \verb=C-h= ini juga dapat kita ganti dengan tombol \verb=F1=, jadi 
kalau mau melihat dokumentasi tentang \emph{mode} dokumen yang sedang
digunakan, cukup tekan \verb=F1 m=, yang \emph{equal} dengan \verb=C-h m=.

\subsection{Suspend Mode}
Fitur \emph{suspend mode} ini berguna ketika Anda tidak sedang dalam 
\emph{mode} \verb=GUI=, yakni untuk beralih dengan cepat ke 
\emph{Shell Interpreter} kemudian menjalankan \emph{Shell command} dari sana.

Caranya adalah:

\begin{verbatim}
|----------+---------------|
| Perintah | Keterangan    |
|----------+---------------|
| C-z      | Suspend emacs |
|----------+---------------|
\end{verbatim}

Ketika Anda sedang dalam \emph{mode} \verb=GUI=, dan menekan \verb=C-z=, maka
layar \verb=emacs= akan ter-\emph{minimize} (\emph{iconify}).

Begitu sudah selesai, Anda dapat kembali ke \verb=emacs= dengan mengetik
\verb=fg= pada \emph{Shell}.

\begin{verbatim}
|----------+------------------|
| Perintah | Keterangan       |
|----------+------------------|
| fg       | kembali ke emacs |
|----------+------------------|
\end{verbatim}

\subsection{Undo dan Redo}
Untuk melakukan \emph{undo} dan \emph{redo} pada \verb=emacs=:

\begin{verbatim}
|----------+-------------------------------------------|
| Perintah | Keterangan                                |
|----------+-------------------------------------------|
| C-x u    | Undo                                      |
| C-_      | Undo                                      |
| C-g C-_  | Redo, untuk multiple redo, tekan C-_ lagi |
|          | sampai menemukan yang dicari.             |
|----------+-------------------------------------------|
\end{verbatim}

Konsep \emph{redo} pada \verb=emacs= agak membingungkan, saran saya banyak
banyaklah berlatih dan membiasaka diri dengan \emph{redo} pada \verb=emacs=.

\subsection{Transpose}

Sesuai dengan namanya, fungsi perintah ini adalah untuk mengubah urutan
karakter / kata, sebagai contoh:

\begin{verbatim}
thisi s
\end{verbatim}

Letakkan \emph{point} pada spasi, kemudian tekan:

\begin{verbatim}
C-t
\end{verbatim}

Dan hasilnya:

\begin{verbatim}
this is
\end{verbatim}

Sekarang menggunakan kalimat \verb=this is= di atas, mari kita rubah
menjadi \verb=is this=:

Letakkan \emph{point} pada kata \verb=this=, kemudian tekan:

\begin{verbatim}
M-t
\end{verbatim}

Hasilnya:

\begin{verbatim}
is this
\end{verbatim}

Anda juga dapat melakukan \emph{transpose} pada baris di mana \emph{point}
berada dengan baris di atasnya:

\begin{verbatim}
C-x C-t
   : transpose baris dengan baris atasnya
\end{verbatim}

Dan berikut adalah rangkuman perintah untuk \emph{transpose}:

\begin{verbatim}
|----------+-----------------------------------------|
| Perintah | Keterangan                              |
|----------+-----------------------------------------|
| C-t      | transpose point dengan point sebelumnya |
| M-t      | transpose kata dengan kata sebelumnya   |
| C-x C-t  | transpose baris dengan baris sebelumnya |
|----------+-----------------------------------------|
\end{verbatim}

\subsection{Rectangles / Multi-Cursor}
Fitur ini berfungsi untuk menyunting kolom pada teks. Sebagai contoh Anda 
memiliki kode \verb=javascript= berikut:

\begin{verbatim}
var a = 'me';
var b = 'you';
var c = 'they';
\end{verbatim}

Anda ingin mengubah kata \verb=var= menjadi \verb=this.=. Dengan menggunakan
fitur \verb=rectangles=, hal ini mudah untuk dilakukan. Berikut caranya:

\begin{verbatim}
|---------------+-------------------------------------------|
| Perintah      | Keterangan                                |
|---------------+-------------------------------------------|
| C-n           | Turun satu baris                          |
| C-n           | Turun satu baris                          |
| M-f           | Menuju ke kata selanjutnya                |
| M-f           | Menuju ke kata selanjutnya                |
| C-b           | Menuju ke 1 karakter sebelumnya           |
| C-x r t this. | Ganti rectangle content dengan kata this. |
| Enter         | Enter                                     |
|---------------+-------------------------------------------|
\end{verbatim}

Perintah \verb=rectangle= yang lain:

\begin{verbatim}
|---------------------------------------------+--------------------------------|
| Perintah                                    | Keterangan                     |
|---------------------------------------------+--------------------------------|
| C-x r k                                     | hapus (kill) teks              |
|                                             | dalam region yang disorot      |
| C-x r d                                     | hapus (delete) teks dalam      |
|                                             | region yang disorot            |
| C-x r y                                     | tempel (yank) rectangle        |
|                                             | terakhir yang dihapus (kill)   |
| C-x r o                                     | insert blank space ke dalam    |
|                                             | region yang disorot            |
| C-x r N                                     | memasukkan nomor baris pada    |
|                                             | region yang disorot            |
| C-x r c                                     | mengosongkan (clear) region    |
|                                             | yang disorot (ganti dengan)    |
| M-x delete-whitespace-rectangle             | spasi menghapus (delete) semua |
|                                             | whitepace dalam region         |
| C-x r t string <RET>                        | replace region yang disorot    |
|                                             | dengan string teks             |
| M-x string-insert-rectangle<RET>string<RET> | memasukkan teks string dalam   |
|                                             | setiap baris pada rectangle    |
|---------------------------------------------+--------------------------------|
\end{verbatim}

Perbedaan antara \emph{kill} dan \emph{delete} adalah kalau perintah
\emph{kill}, Anda menyimpan teks dalam \emph{rectangle ring} 
(\emph{clipboard}), sedangkan perintah \emph{delete} tidak menyimpan.

Satu contoh lagi penggunaan \verb=C-x r c=, misal Anda memiliki teks berikut:

\begin{verbatim}
++++++++++
++++++++++
++++++++++
++++++++++
++++++++++
\end{verbatim}

Lakukan perintah berikut (\emph{point} berada di pojok kiri atas)

\begin{verbatim}
|-----------+-----------------------------------------------|
| Perintah  | Keterangan                                    |
|-----------+-----------------------------------------------|
| C-a       | Menuju ke awal baris                          |
| C-n       | Menuju ke baris ke-2                          |
| C-f       | Maju 1 karakter                               |
| C-Space   | Mart set                                      |
| C-u 2 C-n | Turun 2 baris                                 |
| C-e       | Menuju ke akhir baris                         |
| C-b       | Mundur 1 karakter                             |
| C-x r c   | rectangle clear (ganti karakter dengan spasi) |
|-----------+-----------------------------------------------|
\end{verbatim}

Hasilnya:

\begin{verbatim}
++++++++++
+        +
+        +
+        +
++++++++++
\end{verbatim}

\emph{Pretty neat, huh?}

\subsection{Melakukan Perulangan Angka}
Pada bagian atas sudah disinggung bagaimana melakukan perulangan pada 
\verb=emacs=, yakni dengan menggunakan perintah:

\begin{verbatim}
C-u 5 $
\end{verbatim}

Maka hasilnya:

\begin{verbatim}
$$$$$
\end{verbatim}

Namun perulangan ini tidak berlaku untuk angka, misal Anda ingin mengulang
angka 8 sebanyak 8 kali, maka perintah \verb=C-u 8 8= tidak akan berhasil.
Bagaimana caranya:

\begin{verbatim}
C-u 8 C-u 8
\end{verbatim}

Dan, hasilnya:

\begin{verbatim}
88888888
\end{verbatim}

\subsection{Dynamic Abbrev}
Fitur ini mungkin bagi sebagian orang dinamakan sebagai \emph{auto completion},
sesuai dengan namanya, fitur ini berfungsi untuk \emph{auto complete} dengan 
cara mencocokkan dengan pola kata yang sudah ada pada dokumen.

Caranya adalah dengan menekan:

\begin{verbatim}
|----------+------------------------------------------|
| Perintah | Keterangan                               |
|----------+------------------------------------------|
| M-/      | auto-complete word                       |
| C-M-/    | menampilkan semua kata yang cocok        |
|          | (urut sesuai dengan frekuensi kemunculan |
|          | dalam dokumen                            |
|----------+------------------------------------------|
\end{verbatim}

Misal dalam dokumen Anda sudah terdapat kata \verb=emacs=, maka Anda dapat
melakukan \emph{auto-complete} dengan menekan:

\begin{verbatim}
em M-/
\end{verbatim}

Berikutnya akan muncul kata yang diawali dengan kata \verb=em=. Tekan 
\verb=M-/= lagi akan membawa Anda dengan kata lain yang cocok dengan pola
tersebut.

\subsection{Hippie-expand}
Jika pada \verb=dabbrev-expand=, Anda dapat melakukan \emph{auto-completion}
pada kata, maka dengan \verb=hippie-expand=, Anda dapat mengulang baris.

Caranya adalah dengan mengetikkan beberapa karakter baris yang ingin Anda 
lakukan \emph{auto-completion}, kemudian tekan:

\begin{verbatim}
|-------------------+------------------------------|
| Perintah          | Keterangan                   |
|-------------------+------------------------------|
| M-x hippie-expand | Mengaktifkan line-completion |
|-------------------+------------------------------|
\end{verbatim}

Maka secara otomatis, \verb=emacs= akan melakukan \emph{auto-completion}
untuk baris yang memiliki pola yang cocok dengan karakter-karakter tersebut.


\subsection{FTP atau SCP Berkas pada Dired Mode}
Anda dapat melakukan perintah \verb=ftp= atau \verb=scp= pada \emph{dired-mode}
dengan cara:

\begin{verbatim}
Mark the files
! scp * host:/target/dir/ & <RETURN>
\end{verbatim}

Di sini \verb=emacs= akan mengganti tanda \verb=*= dengan berkas yang sudah
ditandai dan tanda \verb=&= berarti perintah akan dijalankan secara 
\emph{async}.

\subsection{Menuju ke Baris Tertentu}
Kita dapat menuju ke baris tertentu pada dokumen dengan menggunakan perintah:

\begin{verbatim}
M-g g <Number> <RETURN>
\end{verbatim}

Sebagai contoh Anda ingin menuju ke baris 1, maka perintahnya:

\begin{verbatim}
M-g g 1 <RETURN>
\end{verbatim}

\subsection{Membuat Tabel}
Secara \emph{default}, \verb=emacs= memiliki fitur untuk membuat tabel dalam
format \verb=ASCII=. Bagaimana caranya:

\begin{verbatim}
M-x table-insert
\end{verbatim}

Selanjutnya akan muncul pertanyaan (konfirmasi) mengenai berapa jumlah kolom, 
baris dan pertanyaan lain terkait dengan tabel yang ingin Anda buat.

Kemudian Anda tinggal masukkan data ke dalam kolom dan baris yang tersedia.
Untuk berpindah ke kolom (baris) berikutnya, tekan tombol \verb=Tab=.

Berikut ini contoh tabel \verb=ASCII= dalam \verb=emacs=:

\begin{verbatim}
+-----+------+------------------+
|Nama |Alamat|Jenis Kelamin     |
|     |      |                  |
+-----+------+------------------+
|     |      |                  |
+-----+------+------------------+
|     |      |                  |
+-----+------+------------------+
|     |      |                  |
+-----+------+------------------+
|     |      |                  |
+-----+------+------------------+
\end{verbatim}

Untuk mengetahui perintah-perintah lain terkait dengan tabel, ketik:

\begin{verbatim}
M-x table-<Tab>
\end{verbatim}

\subsection{Membuat Tabel Latex}
Bagi Anda yang sering berkutat dengan \verb=LaTex=, tentu tidak asing lagi
dengan \emph{syntax} pembuatan tabel yang terasa membingungkan dan sulit
dihafal.

Untunglah \verb=emacs= menyediakan \emph{tools} di mana kita bisa membuat
tabel di \verb=emacs= untuk kemudian di-\emph{export} ke \emph{syntax}
\verb=LaTex=.

Pertama kita buat dulu tabel pada \verb=emacs= dengan cara:

\begin{verbatim}
M-x table-insert
Masukkan variabel jumlah kolom, baris, dan seterusnya
\end{verbatim}

Yang hasilnya kurang lebih seperti berikut ini:

\begin{verbatim}
+-----+-----+-----+
|     |     |     |
+-----+-----+-----+
|     |     |     |
+-----+-----+-----+
|     |     |     |
+-----+-----+-----+
\end{verbatim}

Anda dapat berpindah dari satu kolom ke kolom berikutnya dengan menekan
tombol \verb=Tab=. Dengan \emph{point} masih berada pada tabel, mari kita ubah 
tabel tersebut ke dalam bahasa \verb=LaTex= dengan cara:

\begin{verbatim}
C-^
latex <RETURN>
\end{verbatim}

Berikutnya akan muncul \emph{window} baru berisikan kode \verb=LaTex= untuk 
tabel kita tadi. Anda tinggal menyalin kode tersebut ke dalam dokumen Anda.
Selesai!

Berikut ini contoh kode untuk tabel kita di atas:

\begin{verbatim}
% This LaTeX table template is generated by emacs 23.3.1
\begin{tabular}{|l|l|l|}
\hline
Number & Title & Address \\
\hline
 & & \\
\hline
 & & \\
\hline
\end{tabular}
\end{verbatim}

\subsection{Keyboard Macro}
Ini adalah fitur yang mampu meningkatkan efektivitas ketika bekerja dengan teks.
Secara \emph{default}, berikut cara untuk bekerja dengan \emph{macro}:

\begin{verbatim}
|----------+-----------------|
| Perintah | Keterangan      |
|----------+-----------------|
| F3       | emulai macro    |
| F4       | engakhiri macro |
| F4       | engulang macro  |
|----------+-----------------|
\end{verbatim}

Selain tombol di atas, kita juga dapat menggunakan tombol berikut:

\begin{verbatim}
|----------+------------------|
| Perintah | Keterangan       |
|----------+------------------|
| C-x (    | memulai macro    |
| C-x )    | mengakhiri macro |
| C-x e    | eksekusi macro   |
|----------+------------------|
\end{verbatim}

Sebagai contoh, ketika bekerja dengan \verb=LaTex=, Anda akan sering sekali
mengetik \verb=\emph{}= untuk mencetak miring sebuah kata.

Untuk menghindari pengetikan berulang-ulang, Anda dapat menyimpan hal tersebut 
ke dalam sebuah \emph{macro}:

\begin{verbatim}
|----------+---------------------------------|
| Perintah | Keterangan                      |
|----------+---------------------------------|
| F3       | memulai macro                   |
| \emph{ } | tulis teks yang akan Anda ulang |
| F4       | akhiri macro                    |
|----------+---------------------------------|
\end{verbatim}

Sekarang waktunya untuk mengulang \emph{macro} tersebut, caranya adalah cukup
dengan menekan \verb=F4= sebanyak yang Anda suka, dan teks tersebut akan
muncul secara otomatis.

Anda juga dapat memberi nama pada \emph{macro} yang sudah Anda buat
sebelumnya. Pada contoh di atas, kita beri nama \texttt{miring}. Caranya:

\begin{verbatim}
|------------------------------------+--------------|
| Perintah                           | Keterangan   |
|------------------------------------+--------------|
| C-x C-k n (kmacro-name-last-macro) | Memberi nama |
| miring                             | nama         |
| <RET>                              |              |
|------------------------------------+--------------|
\end{verbatim}

Sekarang Anda dapat memanggil \emph{macro} dengan menekan tombol
\texttt{M-x miring}, dan teks di atas akan terpanggil.

Nama ini akan tersimpan selama session masih aktif, jika Anda ingin
permanen, Anda dapat menyimpannya dalam sebuah berkas tersendiri atau
cukup tambahkan dalam berkas \texttt{~/.emacs} Anda.

Caranya buka berkas \texttt{~/.emacs} kemudian tambahkan pada baris
paling bawah, dengan menekan:

\begin{verbatim}
M-x insert-kbd-macro <RET> miring <RET>
\end{verbatim}

Sekarang \emph{macro} sudah tersimpan secara permanen, Anda dapat
memanggil \emph{macro} ini dengan menekan tombol

\begin{verbatim}
M-x miring
\end{verbatim}

\subsection{Lowercase, Uppercase dan Capitalize}
Berikut ini perintah untuk mengubah teks menjadi \emph{uppercase/lowercase}
pada \verb=emacs=:

\begin{verbatim}
|----------+------------------------------------------------|
| Perintah | Keterangan                                     |
|----------+------------------------------------------------|
| C-x C-u  | uppercase                                      |
| C-x C-l  | lowercase                                      |
| M-l      | membuat lowercase dari point sampai akhir kata |
| M-u      | membuat uppercase dari point sampai akhir kata |
| M-c      | membuat capitalize daerah yang disorot         |
|----------+------------------------------------------------|
\end{verbatim}

\subsection{Mengaktifkan Mode Baca (Read-Only)}
Dengan mengaktifkan status \emph{read-only}, maka kita tidak dapat menyunting
berkas yang ada pada \verb=emacs=. Hal ini penting untuk menghindari adanya
ketikan-ketikan dari pihak luar yang tidak kita inginkan.

Berikut perintah untuk menyalakan / mematikan status \emph{read-only}:

\begin{verbatim}
|----------+-----------------------------------------|
| Perintah | Keterangan                              |
|----------+-----------------------------------------|
| C-x C-q  | Menyalakan / mematikan status read-only |
|----------+-----------------------------------------|
\end{verbatim}

Mode \emph{read-only} yang aktif ditandai dengan tanda \verb=%%= di bagian 
bawah dari layar \verb=emacs= Anda.

\subsection{Memahami Mode Layar pada emacs}

Jika Anda perhatikan pada bagian bawah dari layar kerja \verb=emacs= Anda,
ada beberapa tanda yang penting untuk mengetahui status dari pekerjaan 
yang sedang kita kerjakan.

\verb=-UU-:**--F1=

\verb=-UU-:----F1=

\verb=-UU-:%%--F1=

Tanda \verb=**= berarti ada pekerjaan yang belum kita simpan, sedangkan tanda
\verb=--= berarti pekerjaan belum mengalami perubahan sejak terakhir kali
kita melakukan penyimpanan, dan tanda \verb=%%= berarti kita sedang berada
pada modus \emph{read-only}.

\subsection{Menjalankan IRC Client}
Anda mungkin masih ingat dengan \emph{irc}? Sebuah layanan \emph{chat}
berbasis teks yang dulu sempat berjaya (waktu itu belum ada Facebook,
Twitter, dll). Dan tahukah Anda bahwa layanan itu sampai sekarang masih
\emph{eksis} meski tidak seramai dulu. Salah satu yang masih aktif
menggunakan layanan ini adalah komunitas \emph{geek} yang bertebaran
di berbagai \emph{room}, termasuk komunitas pengembangan Python yang
dapat Anda temui di \verb=#python=. 

Anda tentu juga tahu program \verb=mirc=? Sebuah program \emph{irc chat client}
yang ngetrend sekali pada jamannya. Syukurlah pengguna \verb=Emacs=, Anda 
tidak perlu untuk memasang \emph{client} tersendiri, karena \verb=Emacs= 
sudah memasukkan program untuk itu, namanya \verb=erc=.

Anda dapat mengaktifkan \verb=erc= dengan cara:

\begin{verbatim}
M-x erc
\end{verbatim}

Selanjutnya tinggal ikuti langkah-langkah seperti Anda menggunakan program
\emph{irc client} lainnya. Untuk dapat bergabung dalam komunitas
ini Anda harus memiliki \emph{id} yang sudah teregistrasi pada 
\verb=NickServ=, jika tidak Anda akan di-\emph{redirect} ke ruang
\verb=#python-unregistered=.

Selamat mencoba!

\subsection{Dari org ke tex dan PDF}
Pada bahasan di atas sempat disinggung bagaimana mengubah dokumen \verb=Org=
ke dalam format \verb=HTML=. Ketika Anda menekan tombol \verb=C-cC-e= 
kalau Anda perhatikan banyak sekali jenis dokumen yang didukung 
oleh \verb=emacs=. Salah satu favorit saya adalah format \verb=tex=. 
Anda dapat langsung memilih format \verb=PDF= atau \verb=tex= saja. 
Kalau ingin praktis, format \verb=PDF= mungkin jadi pilihan, tapi kalau 
ingin lebih fleksibel, format \verb=tex= tentu tidak ada tandingan.

Kenapa fleksibel? Karena jika menggunakan format langsung ke \verb=PDF=,
semuanya serba otomatis, mulai dari judul dokumen yang menggunakan nama
berkas, tanggal, warna halaman isi, pengarang, jenis \emph{font}, dan 
masih banyak lagi. Berdasarkan pengalaman, hasil konversi ke \verb=PDF=
ini kurang begitu bagus.

Oleh karena itu penulis lebih memilih mengubah dokumen \verb=org= ke dalam
dokumen \verb=tex=, baru setelah itu kita ubah secara manual menggunakan
\emph{engine} \verb=pdflatex=.

Bagaimana caranya? Untuk ekspor dokumen, seperti sudah dibahas pada bagian
sebelumnya, tekan \verb=C-cC-e= kemudian tekan \verb=l= untuk format 
\verb=tex=. Setelah itu silakan sunting seperlunya, kalau penulis biasanya
menghilangkan ukuran huruf yang secara \emph{default} ditetapkan 11pt,
kemudian menonaktifkan paket \verb=fontenc= agar tampilan huruf terlihat
lebih halus.

\begin{verbatim}
|-------------------------------+---------------------------|
| Dari                          | Menjadi                   |
|-------------------------------+---------------------------|
| \documentclass[11pt]{article} | \documentclass{article}   |
| \usepackage[T1]{fontenc}      | %\usepackage[T1]{fontenc} |
|-------------------------------+---------------------------|
\end{verbatim}

Jika sudah, sekarang saatnya mengubah dari format \verb=tex= menjadi
\verb=PDF= dengan menekan tombol \verb=C-cC-c= kemudian tekan \emph{Enter}.
Dalam sekejap \verb=emacs= membuat berkas baru dengan ekstensi \verb=PDF= pada
direktori yang sama dengan direktori tempat berkas \verb=tex= berada.

Untuk melihat hasil, Anda dapat menekan tombol \verb=C-cC-c= sekali lagi, 
dan perhatikan layar \verb=emacs= Anda berubah menjadi \emph{pdf viewer}. Atau
Anda dapat juga menggunakan aplikasi eksternal untuk membuka berkas \verb=PDF=
tersebut.

\subsection{Memanggil Menubar pada emacs-nox}
Jika Anda bekerja dengan mode GUI, tentu tidak jadi soal, karena Anda dapat
memanggil/membuka menubar menggunakan perangkat tetikus Anda, tapi jika
Anda bekerja pada emacs-nox, bagaimana cara memanggil menubar? 

Ternyata caranya cukup mudah, hanya dengan menekan tombol \verb=F10= saja.

\begin{verbatim}
|----------+-------------------------------------------|
| Perintah | Keterangan                                |
|----------+-------------------------------------------|
| F10      | Memanggil / Membuka Menubar pada mode nox |
|----------+-------------------------------------------|
\end{verbatim}

\subsection{Import Tabel HTML kedalam Org-Mode}
Untuk mengimport html table ke dalam orgmode, salin dan tempel tabelnya dulu, kemudian tekan 
\verb=C-c |=

\subsection{Access Command Line Manual (man)}
Misal Anda ingin mengetahui manual untuk perintah ls, tanpa harus keluar dari
emacs Anda dapat membacanya dengan:

\begin{verbatim}
|--------------+-----------------------|
| Perintah     | Keterangan            |
|--------------+-----------------------|
| M-x man      | Mengaktifkan menu man |
| command-line | misal, ls, tr, dll.   |
|--------------+-----------------------|
\end{verbatim}

\subsection{Access Info Mode}
\emph{Info-mode} tidak lain adalah sumber informasi untuk emacs sekaligus beberapa topik yang masih berkaitan, khususnya terkait dengan sistem operasi Unix 
(CMIIW), sangat disarankan untuk dibaca, baik bagi Anda yang baru mencoba 
sistem operasi emacs dan Unix (atau turunannya) maupun Anda yang sehari-hari 
sudah bergelut dengan emacs/Unix.

\begin{verbatim}
|----------+-----------------------------|
| Perintah | Keterangan                  |
|----------+-----------------------------|
| M-x info | Mengakses dokumentasi info  |
|----------+-----------------------------|
\end{verbatim}

Untuk melakukan navigasi pada mode ini, Anda dapat menekan tombol \verb=F10=
kemudian tekan \verb=i= untuk \emph{info}, dari sini Anda dapat memilih
navigasi yang Anda inginkan (\emph{up, next}, dll.)

atau gunakan \emph{shortkey} berikut:

\begin{verbatim}
|----------+------------------|
| Perintah | Keterangan       |
|----------+------------------|
| h        | memunculkan help |
| u        | up               |
| n        | next             |
| p        | previous         |
| Ret      | jump into link   |
|----------+------------------|
\end{verbatim}

\subsection{Melakukan Kustomisasi}
Salah satu kelebihan emacs adalah pada ekstensibilitas yang
dimilikinya. Pengguna dapat dengan mudah melakukan kustomisasi dengan
mengaktifkan mode customization (\verb=M-x customize=) dan melakukan
pencarian pada kotak pencarian yang disediakan.

Sebagai contoh kita ingin melakukan kustomisasi terhadap
\verb=ido-mode=, maka cari ido-mode, kemudian ubah pengaturan dari
sana. Setelah itu kita dapat menyimpan pengaturan dengan memilih
tautan state, dan pilih model penyimpanan, apakah hanya untuk sesi
sekarang, atau disimpan untuk sesi-sesi yang akan datang, jika pilihan
kedua, maka emacs akan secara otomatis menyunting berkas \verb=.emacs=
pada direktori home Anda.

\subsection{Merapikan Baris}
Untuk membuat kegiatan sunting-menyunting berkas dengan \verb=emacs=
lebih rapi dan enak dilihat, saya biasa membatasi panjang setiap baris
maksimal 78 karakter, biar tampilan emacs menjadi rapi dan tidak terlalu
lebar.

Caranya mudah, cukup menekan \verb=M-q=, dan otomatis baris yang panjang
akan terpotong sesuai dengan pengaturan yang Anda tetapkan (secara 
\emph{default} maksimal 78 karakter.


\subsection{Lain-lain}

\begin{verbatim}
|--------------------------------+---------------------------------------------|
| Perintah                       | Keterangan                                  |
|--------------------------------+---------------------------------------------|
| M-\                            | remove whitespace before and after cursor   |
|                                | and after cursor                            |
| M-^                            | join the line with previous line            |
|                                | and fix white space                         |
| M-x delete-trailing-whitespace | removes blanks after last char on all lines |
|--------------------------------+---------------------------------------------|
\end{verbatim}

Dalam github ini juga disertakan contoh konfigurasi berkas
\texttt{.emacs} yang penulis gunakan.

\url{https://github.com/kholidfu/emacs\_doc}

\section{Version History}
Bagian ini merupakan \emph{log} dari perubahan-perubahan yang ada pada buku
kecil ini.

\begin{verbatim}
|------------------+-------+----------------------------------------|
| Tanggal          | Versi | Keterangan                             |
|------------------+-------+----------------------------------------|
| 17 Januari 2013  |   0.1 | Versi pertama                          |
| 25 April 2013    |   0.2 | Versi kedua, menambahkan beberapa      |
|                  |       | subsection baru, seperti konversi      |
|                  |       | dari org ke tex kemudian ke pdf        |
| 29 Desember 2014 |   0.3 | Versi ketiga, sekarang dokumentasi ini |
|                  |       | available di github                    |
|------------------+-------+----------------------------------------|
\end{verbatim}

\end{document}
